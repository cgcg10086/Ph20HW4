\documentclass{article}

% Packages
\usepackage[top=1in, left=0.75in, right=0.75in, bottom=1in]{geometry}
\usepackage{amsmath}
\usepackage{amssymb}
\usepackage{array}
\usepackage{fancyhdr}
\usepackage{color}
\usepackage{graphicx}
\usepackage{verbatim}
\usepackage{pbox}
\usepackage{amsfonts}
\usepackage{listings}

\newcommand{\9}{\left( }
\newcommand{\0}{\right) }

\begin{document}
\pagestyle{fancy}
\lhead{Chen Chang}
\chead{Ph 20 HW4}
\rhead{May 14, 2018}

\noindent \textbf{Part 1}
\begin{enumerate}
    \item Initial conditions $x_0 = 0, v_0 = 1$. Using $h = 0.01$ and $N = 1500$.\\
    \includegraphics[scale=0.5]{explicit_xv.png}
    
    \item For the simple mass-on-a-spring oscillator, $F = -kx = ma = m\frac{dx^2}{dt}$, so $\frac{dx^2}{dt} = -x$ since we use $k/m = 1$. It's been years since I've seen a derivative (let along a diffeq), so I dug up an old textbook and found that the solution to this is $x(t) = A \cos(t) + B\sin(t)$, which gives $v(t) = \frac{dx}{dt} = -A\sin(t) + B\cos(t)$ (thank you WolframAlpha, I'm not kidding when I say it's been years). Then for the initial conditions $x(0) = 0$ and $v(0) = 1$ from part 1, we have $0 = A$ and $1 = B$, so the solution is $x(t) = \sin(t)$ and $v(t) = \cos(t)$. Plotting $analytic - explicit$ as the errors, we get:\\
    \includegraphics[scale=0.5]{explicit_xv_error.png}
    
    \item Plotting maximum magnitude error versus various scaling factors for $h$ (keeping all other initial conditions the same), we see that the relation between $h$ and the error is indeed linear:\\
    \includegraphics[scale=0.5]{explicit_error_h.png}
    
    \item Plotting the normalized total energy, we see that overtime, the total energy increases over time, which agrees with how the global error increases over time.\\
    \includegraphics[scale=0.5]{explicit_energy.png}
    
    \item Multiplying the matrix in (9), we get $x_{i+1} - h \cdot v_{i+1} = x_i$ and $h\cdot x_{i+1} + v_{i+1} = v_i$.  We can solve this system for $x_{i+1}$ and $v_{i+1}$ to get  $h x_{i+1} - h^2 v_{i+1} = h x_i \implies (v_i - v_{i+1}) - h^2 v_{i+1} = h x_i \implies (h^2 + 1)v_{i+1} = v_i - hx_i$ so $v_{i+1} = \frac{v_i - hx_i}{h^2 + 1}$. Similarly, $h^2 x_{i+1} + hv_{i+1} = hv_i \implies h^2 x_{i+1} + (x_{i+1} - x_i) = hv_i \implies (h^2 + 1)x_{i+1} = hv_i + x_i$ so $x_{i+1} = \frac{hv_i + x_i}{h^2 + 1}$. Computing and plotting the error versus h and total energy for this implicit method, we see that the global error increases in magnitude over time, same as the explicit Euler method. The implicit method's error is also proportional to $h$, but has positive slope. This agrees with the observation that the total energy decreases over time, which would cause $x_{analytic} - x_i$ to be positive.\\
    \includegraphics[scale=0.5]{implicit_xv_error.png}\\
    \includegraphics[scale=0.5]{implicit_error_h.png}\\
    \includegraphics[scale=0.5]{implicit_energy.png}
    
\end{enumerate}


\noindent \textbf{Part 2}
\begin{enumerate}
    \item We observe that the explicit method spirals outward from the analytic solution (circle), reflecting the increase in energy over time. Similarly, the implicit method spirals inward, reflecting the decrease in energy over time.\\
    \includegraphics[scale=0.5]{imp_exp_real_phasediagram.png}
    
    \item The symplectic method also forms a closed loop, and practically overlaps with the analytic solution. There are a couple regions where it is, with careful observation at high zoom levels, visibly not identical (slightly inside or outside). We don't include the explicit and implicit methods on this plot to better observe how the symplectic method relates to the analytical solution.\\
    \includegraphics[scale=0.5]{symp_phasediagram.png}
    
    \item The total energy for the symplectic solution oscillates about 1. This reflects how the phase-space plot is a closed loop rather than a continuous inward or outward spiral, with the oscillating error representing how it is not exactly the same as the analytic solution. The total energy is not perfectly accurate, but it doesn't exceed a maximum deviation.\\
    \includegraphics[scale=0.5]{symp_energy.png}
    
    
\end{enumerate}
\noindent \textbf{Source Code}
\lstinputlisting[language=Python]{diffeq.py}

\noindent \textbf{Makefile}
\lstinputlisting[language=make]{Makefile}

\noindent \textbf{Git Log}
\lstinputlisting{git.log}
\end{document}