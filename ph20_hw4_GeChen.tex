% This is how you make a comment
\documentclass[8pt]{article}
% For pictures
\usepackage{graphicx}
\usepackage{float} % this can force a figure/table to stay where you want. 
% For math
\usepackage{amssymb}
\usepackage{amsmath}
% To include code 
\usepackage{listings}
\usepackage{color}

\definecolor{codegreen}{rgb}{0,0.6,0}
\definecolor{codegray}{rgb}{0.5,0.5,0.5}
\definecolor{codepurple}{rgb}{0.58,0,0.82}
\definecolor{backcolour}{rgb}{0.95,0.95,0.92}
 
\lstdefinestyle{mystyle}{
    backgroundcolor=\color{backcolour},   
    commentstyle=\color{codegreen},
    keywordstyle=\color{magenta},
    numberstyle=\tiny\color{codegray},
    stringstyle=\color{codepurple},
    basicstyle=\footnotesize,
    breakatwhitespace=false,         
    breaklines=true,                 
    captionpos=b,                    
    keepspaces=true,                 
    numbers=left,                    
    numbersep=5pt,                  
    showspaces=false,                
    showstringspaces=false,
    showtabs=false,                  
    tabsize=2
}
 
\lstset{style=mystyle}

\begin{document}
% Title
\title{Assignment 4}
\author{Ge Chen} 
\maketitle
% Sections and text
\section{Part 1} 

This section investigates the implicit and explicit Euler methods. Please see figures. 

\begin{figure}[H] % force the figure to stay HERE. 
\centering
\includegraphics[width=\textwidth]{Fig1ExplicitEuler.pdf}
\caption{$f_{x}:f_{y}=1:2$. There is 1 horizontal lobe and 2 vertical lobes.}
\label{fig:1-to-2}
\end{figure} 

\begin{figure}[H]
\centering
\includegraphics[width=\textwidth]{Fig2ErrorExplicitEuler.pdf}
\caption{$f_{x}:f_{y}=2:4$ produces the same curve with that of $f_{x}:f_{y}=1:2$} 
\label{fig:2-to-4}
\end{figure} 

\begin{figure}[H]
\centering
\includegraphics[width=\textwidth]{Fig3Error_h.pdf}
\caption{$f_{x}:f_{y}=2:3$ creates 2 horizontal lobes and 3 vertical lobes.} 
\label{fig:2-to-3}
\end{figure} 

\begin{figure}[H]
\centering
\includegraphics[width=\textwidth]{Fig4EnergyEvo.pdf}
\caption{$f_{x}:f_{y}=4:5$ creates 4 horizontal lobes and 5 vertical lobes.} 
\label{fig:4-to-5}
\end{figure} 

\begin{figure}[H]
\centering
\includegraphics[width=\textwidth]{Fig5ImplicitEuler.pdf}
\caption{$f_{x}:f_{y}=2:3$ creates 3 horizontal lobes and 1 vertical lobe.} 
\label{fig:3-to-1}
\end{figure} 

\section{Part 2} 
This section investigates the phase space behaviors of many methods. 

\begin{figure}[H]
\centering
\includegraphics[width=\textwidth]{Part2Fig6PhaseSpace.pdf}
\caption{$\phi = 0$ produces a circle.} 
\label{fig:phi0}
\end{figure} 

\begin{figure}[H]
\centering
\includegraphics[width=\textwidth]{Part2Fig7PhaseSpace.pdf}
\caption{$\phi = \frac{1}{4}\pi$ rotates the circle in 3D, and produces a rotated ellipse. } 
\label{fig:phi45}
\end{figure} 

\begin{figure}[H]
\centering
\includegraphics[width=\textwidth]{Part2Fig8EnergyEvoSymp.pdf}
\caption{$\phi = -\frac{1}{4}\pi$ rotates the circle in 3D, and produces an ellipse rotated to the other direction. } 
\label{fig:phi-45}
\end{figure} 

\begin{figure}[H]
\centering
\includegraphics[width=\textwidth]{Part2Fig9PhaseLag.pdf}
\caption{$\phi = \frac{1}{2}\pi$ projects the rotated circle as a line. } 
\label{fig:phi90}
\end{figure} 


\noindent \textbf{Source Code}
\lstinputlisting[language=Python]{hw3_make_fig_GeChen.py} 

\noindent \textbf{Makefile}
\lstinputlisting[language=make]{Makefile_v4.mk}

\noindent \textbf{Git Log}
\lstinputlisting{git.log}


\end{document}
